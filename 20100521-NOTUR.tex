%\documentclass[handout]{beamer}
\documentclass{beamer}

\mode<presentation>
{
%\usetheme{Singapore}
%\usetheme{Warsaw}
\usetheme{Malmoe}
\useinnertheme{circles}
\useoutertheme{infolines}
% \useinnertheme{rounded}

\setbeamercovered{transparent}
}

\usepackage[english]{babel}
\usepackage[latin1]{inputenc}
\usepackage{bm,textpos,alltt,listings,multirow,ulem}

% font definitions, try \usepackage{ae} instead of the following
% three lines if you don't like this look
\usepackage{mathptmx}
\usepackage[scaled=.90]{helvet}
\usepackage{courier}
\usepackage[T1]{fontenc}
\usepackage{tikz}
\usetikzlibrary[shapes.arrows,arrows,shapes.misc]

% \usepackage{pgfpages}
% \pgfpagesuselayout{4 on 1}[a4paper,landscape,border shrink=5mm]

\ProvidesPackage{JedMacros}

\usepackage{amsmath}
\usepackage{multirow,listings,booktabs}
\usepackage{mdwlist}
\usepackage{xspace}
\usepackage[iso,american]{isodate}
\makeatletter
\DeclareRobustCommand\onedot{\futurelet\@let@token\@onedot}
\def\@onedot{\ifx\@let@token.\else.\null\fi\xspace}
\def\eg{{e.g}\onedot} \def\Eg{{E.g}\onedot}
\def\ie{{i.e}\onedot} \def\Ie{{I.e}\onedot}
\def\cf{{cf}\onedot} \def\Cf{{Cf}\onedot}
\def\etc{{etc}\onedot}
\def\vs{{vs}\onedot}
\def\wrt{w.r.t\onedot}
\def\dof{d.o.f\onedot}
\def\etal{{et al}\onedot}
\makeatother

\usepackage{tikz}
\usetikzlibrary[shapes,shapes.arrows,arrows,shapes.misc,fit,positioning]

\usepackage{siunitx}
\DeclareSIUnit\year{a}
\DeclareSIUnit\byte{B}
\sisetup{retain-unity-mantissa = false}

\usepackage{fancyvrb}
\usepackage[outputdir=out]{minted}
\newminted{c}{gobble=2}
\newminted{python}{gobble=2}
%\newmint[cverb]{c}{} 
\newcommand\cverb[1][]{\SaveVerb[%
    aftersave={\textnormal{\UseVerb[#1]{vsave}}}]{vsave}}
\newcommand\cfunc[1][]{\SaveVerb[%
    aftersave={\textnormal{\UseVerb[#1]{vsave}\texttt{()}}}]{vsave}}
\newcommand\pyverb[1][]{\SaveVerb[%
    aftersave={\textnormal{\UseVerb[#1]{vsave}}}]{vsave}}
\def\asm#1{{\tt #1}}
\def\code#1{{\tt #1}}
\def\shell#1{{\tt \$ #1}}

\newcommand\email[1]{{\href{mailto:#1}{\nolinkurl{#1}}}}

\newcommand{\II}{\mathcal{I}}
\newcommand{\C}{\mathbb{C}}
\newcommand{\D}{\mathcal{D}}
\newcommand{\EE}{\mathcal{E}}
\newcommand{\F}{\mathcal{F}}
\newcommand{\I}{\mathcal{I}}
\newcommand{\N}{\mathcal{N}}
\newcommand{\PP}{\mathcal{P}}
\newcommand\Ppc{\ensuremath{\mathsf P}}
\newcommand{\bigO}{\ensuremath{\mathcal{O}}}
\newcommand{\R}{\mathbb{R}}
\newcommand{\Rz}{\mathcal{R}}
\newcommand{\QQ}{\mathcal Q}
\newcommand{\VV}{\mathcal V}
\newcommand{\ASM}{\mathrm{ASM}}
\newcommand{\RASM}{\mathrm{RASM}}

\newcommand{\kb}{\tt}
\newcommand{\Pk}[1]{\ensuremath{P_{#1}}}
\newcommand{\Qk}[1]{\ensuremath{Q_{#1}}}
\newcommand{\Pkdisc}[1]{\ensuremath{P_{#1}^{\text{disc}}}}
\newcommand{\Qkdisc}[1]{\ensuremath{Q_{#1}^{\text{disc}}}}
\newcommand{\blue}{\textcolor{blue}}
\newcommand{\green}{\textcolor{green!70!black}}
\newcommand{\red}{\textcolor{red}}
\newcommand{\brown}{\textcolor{brown}}
\newcommand{\cyan}{\textcolor{cyan}}
\newcommand{\magenta}{\textcolor{magenta}}
\newcommand{\yellow}{\textcolor{yellow}}
\newcommand{\mini}{\mathop{\rm minimize}}
\newcommand{\st}{\mbox{subject to }}
\newcommand{\lap}{\Delta}
\newcommand\mtab{\hspace{\stretch{1}}}
\newcommand\ud{\,\mathrm{d}}
\newcommand\bslash{{$\backslash$}}
\newcommand\half{{\frac 1 2}}
\newcommand{\abs}[1]{\left\lvert #1 \right\rvert}
\newcommand{\bigabs}[1]{\big\lvert #1 \big\rvert}
\newcommand{\norm}[1]{\left\lVert #1 \right\rVert}
\newcommand\oneitem[1]{\begin{itemize} \item #1 \end{itemize}}
\newcommand\pfrak{{\mathfrak p}}
\newcommand\nfrak{{\mathfrak n}}
\newcommand\ff{\bm f}
\newcommand\mm{\bm m}
\newcommand\nn{\bm n}
\newcommand\uu{\bm u}
\newcommand\vv{\bm v}
\newcommand\ww{\bm w}
\newcommand\DD{D}
\newcommand{\tcolon}{\!:\!}
\DeclareMathOperator{\sgn}{sgn}
\DeclareMathOperator{\card}{card}
\DeclareMathOperator{\trace}{tr}
\DeclareMathOperator{\erf}{erf}
\DeclareMathOperator{\sspan}{span}
\DeclareMathOperator{\argmin}{arg\,min}
\renewcommand{\bar}{\overline}
% \DeclareMathOperator{\divergence}{div}
% \renewcommand\div\divergence
\renewcommand{\div}{{\nabla \cdot}}
\newcommand\spliceop{\leftrightsquigarrow}
\newcommand\splice[5]{{#1} \overset{#5}{\underset{#3,#4}{\leftrightsquigarrow}} {#2}}
\newcommand{\ip}[2]{{\left\langle #1, #2 \right\rangle}}
\newcommand{\Linfty}{{L^\infty}}

% Dimensionless numbers
\newcommand{\Peclet}{{\mathrm{Pe}}}
\newcommand{\Reynolds}{{\mathrm{Re}}}
\newcommand{\Rayleigh}{{\mathrm{Ra}}}
\newcommand{\Mach}{{\mathrm{Ma}}}
\newcommand{\Prandtl}{{\mathrm{Pr}}}
\newcommand{\Grashof}{{\mathrm{Gr}}}

\newcommand{\sw}[1]{\textsf{\small #1}}
\newcommand{\PETSc}{\sw{PETSc}\xspace}
\newcommand{\PyClaw}{\sw{PyClaw}\xspace}
\newcommand{\Dohp}{\sw{Dohp}\xspace}
\newcommand\libmesh{\sw{libMesh}\xspace}
\newcommand\dealii{\sw{Deal.II}\xspace}
\newcommand\MatMult{\cverb|MatMult|}
\newcommand\MatSolve{\cverb|MatSolve|}
\newcommand{\secref}[1]{{Section~\ref{#1}}}
\newcommand{\chapref}[1]{{Chapter~\ref{#1}}}
\newcommand{\figref}[1]{{Figure~\ref{#1}}}
\newcommand{\tabref}[1]{{Table~\ref{#1}}}
\newcommand\AIJ{{\cverb|AIJ|}}
\newcommand\AIJInode{\cverb|AIJ|/\cverb|Inode|}
\newcommand\BAIJ[1][]{\ifthenelse{\equal{#1}{}}{\cverb|BAIJ|}{\ensuremath{\cverb|BAIJ|(#1)}}}
\newcommand\SBAIJ[1][]{\ifthenelse{\equal{#1}{}}{\cverb|SBAIJ|}{\ensuremath{\cverb|SBAIJ|(#1)}}}
\newcommand\todo[1]{{\color{red}\bf [TODO: #1]}}
\newcommand\tf[1]{\hat{#1}}     % test functions


\title[PETSc]{The Portable Extensible Toolkit for Scientific computing}

\subtitle{New developments, memory performance, and algorithmic experimentation}

\author{Jed Brown}


% - Use the \inst command only if there are several affiliations.
% - Keep it simple, no one is interested in your street address.
\institute[ETH Z\"urich]
{
  Laboratory of Hydrology, Hydraulics, and Glaciology \\
  ETH Z\"urich
}

\date{NOTUR 2010-05-21, Bergen}

% This is only inserted into the PDF information catalog. Can be left
% out.
\subject{Talks}


% If you have a file called "university-logo-filename.xxx", where xxx
% is a graphic format that can be processed by latex or pdflatex,
% resp., then you can add a logo as follows:

% \pgfdeclareimage[height=0.5cm]{university-logo}{university-logo-filename}
% \logo{\pgfuseimage{university-logo}}



% Delete this, if you do not want the table of contents to pop up at
% the beginning of each subsection:
% \AtBeginSubsection[]
% {
% \begin{frame}<beamer>
% \frametitle{Outline}
% \tableofcontents[currentsection,currentsubsection]
% \end{frame}
% }

% If you wish to uncover everything in a step-wise fashion, uncomment
% the following command:

%\beamerdefaultoverlayspecification{<+->}

\begin{document}
\lstset{language=C}
\normalem

\begin{frame}
\titlepage
\end{frame}

\begin{frame}
\frametitle{Outline}
\tableofcontents
% You might wish to add the option [pausesections]
\end{frame}

\section{Introduction}
\input{slides/PETSc/About.tex}
\newcommand\ganttline[4]{% line, tag, start end
   \node at (0,#1*0.4+.1) [anchor=base east] {#2};
   \fill[blue] (#3/\xtick-1991/\xtick,#1*0.4-.1) rectangle (#4/\xtick-1991/\xtick,#1*0.4+.1);}
\newcommand\ganttlabel[6]{% year, label, color, yloc, anchor
  \node[#3] at (#1/\xtick+#6/\xtick-1991/\xtick,#4) [anchor=#5] {#2};
  \fill[#3] (#1/\xtick-1991/\xtick,1/2-.1) rectangle (#1/\xtick-1991/\xtick+0.04,12/2+.6);}

%\begin{frame}{Timeline}
\begin{frame}[shrink=5]{}
\begin{figure}[htbp]
\def\present{2019.2}
\def\xtick{2.4}
\vspace{-1ex}
\begin{tikzpicture}[y=-1cm]
   %\draw[help lines] (0.5,5) grid (8,0.5);
   \ganttlabel{1991}{1991}{red}{7.5}{north}{0}
   \ganttlabel{1995}{1995}{red}{7.5}{north}{0}
   \ganttlabel{2000}{2000}{red}{7.5}{north}{0}
   \ganttlabel{2005}{2005}{red}{7.5}{north}{0}
   \ganttlabel{2010}{2010}{red}{7.5}{north}{0}
   \ganttlabel{2015}{2015}{red}{7.5}{north}{0}
   \ganttlabel{1992}{PETSc-1}{green!70!black}{0}{center}{0}
   \ganttlabel{1994.4}{MPI-1}{magenta!70!black}{-.5}{center}{0}
   \ganttlabel{1997.6}{MPI-2}{magenta!70!black}{-.5}{center}{0}
   \ganttlabel{1995.5}{PETSc-2}{green!70!black}{0}{center}{0}
   \ganttlabel{2008.9}{PETSc-3}{green!70!black}{0}{center}{0}
   \ganttline{1}{Barry}{1991}{\present}
   \ganttline{2}{Bill}{1991}{1996}
   \ganttline{3}{Lois}{1993}{2001}
   \ganttline{4}{Satish}{1997}{\present}
   \ganttline{5}{Dinesh}{1998}{2005.5}
   \ganttline{6}{Hong}{2001}{\present}
   \ganttline{7}{Kris}{2001}{2006}
   \ganttline{8}{Matt}{2001.5}{\present}
   \ganttline{9}{Victor}{2003}{2006.9}
   \ganttline{9}{}{2007.3}{2007.5}
   \ganttline{9}{}{2008.5}{2008.7}
   \ganttline{10}{Dmitry}{2005.6}{2015.6}
   \ganttline{11}{Lisandro}{2006.9}{\present}
   \ganttline{12}{Jed}{2009}{\present}
   \ganttline{13}{Shri}{2009.8}{\present}
   \ganttline{14}{Peter}{2011.6}{2014.5}
   \ganttline{15}{Mark}{2011.9}{\present}
   \ganttline{16}{Stefano}{2012.1}{\present}
   \ganttline{17}{Toby}{2013.3}{\present}
   \ganttline{18}{Mr. Hong}{2014.7}{\present}
\end{tikzpicture}
\end{figure}
\end{frame}


\section[Memory]{Memory performance for sparse kernels}
\input{slides/JFNKBottlenecks.tex}
\input{slides/CPUArchitecture.tex}
\input{slides/HardwareCapability.tex}
%\input{OlikerSpMVOptimization.tex}
\subsection{Sparse Matrix-Vector products}
\input{slides/SpMVPerformanceModel.tex}
\begin{frame}[shrink=1]{Memory Bandwidth}
%\todo{Replace with performance numbers for current CPUs.}
\begin{itemize}
\item Stream Triad benchmark (GB/s): $\bm w \gets \alpha \bm x + \bm y$
\includegraphics[width=0.8\textwidth]{figures/StreamTriadXT5VsBGP} \\
\item Sparse matrix-vector product: 6 bytes per flop
\includegraphics[width=0.8\textwidth]{figures/SparseMatVec} \\
% {\footnotesize (from Dinesh Kaushik)}
\item Can test on your system using: {\kb cd \$PETSC\_DIR \&\& make streams}
\end{itemize}
\end{frame}

\input{slides/OptimizingSpMV.tex}
\begin{frame}{Optimizing unassembled Mat-Vec}
  \begin{itemize}
  \item High order spatial discretizations do more work per node
    \begin{itemize}
    \item Dense tensor product kernel (like small BLAS3)
    \item Cubic ($Q_3$) elements in 3D can achieve $>70\%$ of peak FPU \\
      (compare to $< 5\%$ for assembled operators on multicore)
    \item Can store Jacobian information at quadrature points \\
      (usually pays off for $Q_2$ and higher in 3D)
    \item Spectral, WENO, DG, FD
    \item Often still need an assembled operator for preconditioning
    \end{itemize}
  \item Boundary element methods
    \begin{itemize}
    \item Dense kernels
    \item Fast Multipole Method (FMM)
    \end{itemize}
  \item<2> \alert{Preconditioning requires more effort}
    \begin{itemize}
    \item Useful to have code to assemble matrices: try out new methods quickly
    \end{itemize}
  \end{itemize}
\end{frame}

\subsection{Triangular solves}
\input{slides/StoringFactors.tex}
\input{slides/FactorImprovement.tex}

\section{Time Integration}
\subsection{Differential Algebraic Equations}
\input{slides/StiffIntegrators.tex}
\input{slides/StiffBarriers.tex}
\input{slides/DAEIRKS.tex}
\subsection{Strong stability preserving methods}
\input{slides/SSPIntegrators.tex}

\section{Preconditioning using splitting methods}
\input{slides/FieldSplit.tex}
\input{slides/SplittingStrongCoupling.tex}

\section{Hydrostatic Ice}
\begin{frame}[shrink=5]{Hydrostatic equations for ice sheet flow}
  \begin{itemize}
  \item Valid when $w_x \ll u_z$, independent of basal friction {\small (Schoof\&Hindmarsh 2010)}
  \item Eliminate $p$ and $w$ from Stokes by incompressibility:\\
    \quad 3D elliptic system for $\bm u = (u,v)$
    \begin{align*}
      - \nabla\cdot \left[ \eta
        \begin{pmatrix}
          4 u_x + 2 v_y & u_y + v_x & u_z \\
          u_y + v_x & 2 u_x + 4 v_y & v_z
        \end{pmatrix} \right] + \rho g \bar\nabla h & = 0
    \end{align*}
    \begin{align*}
      \eta(\theta,\gamma) &= \frac{B(\theta)}{2} (\gamma_0 + \gamma)^{\frac{1-n}{2n}}, \qquad n \approx 3 \\
      \gamma &= u_x^2 + v_y^2 + u_xv_y + \frac 1 4 (u_y+v_x)^2 + \frac 1 4 u_z^2 + \frac 1 4 v_z^2
    \end{align*}
    and slip boundary $\sigma \cdot \bm n = \beta^2 \bm u$ where
    \begin{align*}
      \beta^2(\gamma_b) &= \beta_0^2 (\epsilon_b^2 + \gamma_b)^{\frac{m-1}{2}}, \qquad 0 < m \le 1 \\
      \gamma_b &= \frac 1 2 (u^2 + v^2)
    \end{align*}
  \item $Q_1$ FEM with Newton-Krylov-Multigrid solver in PETSc: \code{src/snes/examples/tutorials/ex48.c}
  \end{itemize}
\end{frame}

\input{slides/THI/X5kmClip.tex}
\input{slides/THI/WhatAboutSplitting.tex}

\input{slides/ThoughtsOnMultigrid.tex}

\begin{frame}{Wrap-up}
  \begin{itemize}
  \item PETSc can help you
    \begin{itemize}
    \item easily construct a code to experiment with ideas
    \item scale an existing code base
    \item incorporate more scalable or higher performance algorithms
    \item attain high performance on a variety of architectures
    \item debug and profile a parallel application (not discussed today)
    \item package and distribute your code (e.g. graph algorithms, \\
      domain decomposition and multilevel solvers), \code{-{}-download-xxx}
    \end{itemize}
  \item I will be around most of today, find me to discuss
    \begin{itemize}
    \item new and old features in PETSc
    \item performance, scalability, and algorithms
    \item design of new codes
    \item integration with your existing application
    \end{itemize}
  \item \url{http://mcs.anl.gov/petsc}
  \item \url{petsc-users@mcs.anl.gov} and \url{petsc-dev@mcs.anl.gov}
  \item \url{petsc-maint@mcs.anl.gov}
  \end{itemize}
\end{frame}

\end{document}
