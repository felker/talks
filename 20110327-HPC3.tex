%\documentclass[handout]{beamer}
\documentclass{beamer}

\mode<presentation>
{
%\usetheme{Singapore}
%\usetheme{Warsaw}
\usetheme{Malmoe}
\useinnertheme{circles}
%\useoutertheme[footline=empty,subsection=false]{miniframes}
\useoutertheme{infolines}
\setbeamercovered{transparent}
}

\usepackage[english]{babel}
\usepackage[latin1]{inputenc}
\usepackage{bm,textpos,alltt,multirow,ulem}

% font definitions, try \usepackage{ae} instead of the following
% three lines if you don't like this look
\usepackage{mathptmx}
\usepackage[scaled=.90]{helvet}
\usepackage{courier}
\usepackage[T1]{fontenc}

% \usepackage{pgfpages}
% \pgfpagesuselayout{4 on 1}[a4paper,landscape,border shrink=5mm]

\ProvidesPackage{JedMacros}

\usepackage{amsmath}
\usepackage{multirow,listings,booktabs}
\usepackage{mdwlist}
\usepackage{xspace}
\usepackage[iso,american]{isodate}
\makeatletter
\DeclareRobustCommand\onedot{\futurelet\@let@token\@onedot}
\def\@onedot{\ifx\@let@token.\else.\null\fi\xspace}
\def\eg{{e.g}\onedot} \def\Eg{{E.g}\onedot}
\def\ie{{i.e}\onedot} \def\Ie{{I.e}\onedot}
\def\cf{{cf}\onedot} \def\Cf{{Cf}\onedot}
\def\etc{{etc}\onedot}
\def\vs{{vs}\onedot}
\def\wrt{w.r.t\onedot}
\def\dof{d.o.f\onedot}
\def\etal{{et al}\onedot}
\makeatother

\usepackage{tikz}
\usetikzlibrary[shapes,shapes.arrows,arrows,shapes.misc,fit,positioning]

\usepackage{siunitx}
\DeclareSIUnit\year{a}
\DeclareSIUnit\byte{B}
\sisetup{retain-unity-mantissa = false}

\usepackage{fancyvrb}
\usepackage[outputdir=out]{minted}
\newminted{c}{gobble=2}
\newminted{python}{gobble=2}
%\newmint[cverb]{c}{} 
\newcommand\cverb[1][]{\SaveVerb[%
    aftersave={\textnormal{\UseVerb[#1]{vsave}}}]{vsave}}
\newcommand\cfunc[1][]{\SaveVerb[%
    aftersave={\textnormal{\UseVerb[#1]{vsave}\texttt{()}}}]{vsave}}
\newcommand\pyverb[1][]{\SaveVerb[%
    aftersave={\textnormal{\UseVerb[#1]{vsave}}}]{vsave}}
\def\asm#1{{\tt #1}}
\def\code#1{{\tt #1}}
\def\shell#1{{\tt \$ #1}}

\newcommand\email[1]{{\href{mailto:#1}{\nolinkurl{#1}}}}

\newcommand{\II}{\mathcal{I}}
\newcommand{\C}{\mathbb{C}}
\newcommand{\D}{\mathcal{D}}
\newcommand{\EE}{\mathcal{E}}
\newcommand{\F}{\mathcal{F}}
\newcommand{\I}{\mathcal{I}}
\newcommand{\N}{\mathcal{N}}
\newcommand{\PP}{\mathcal{P}}
\newcommand\Ppc{\ensuremath{\mathsf P}}
\newcommand{\bigO}{\ensuremath{\mathcal{O}}}
\newcommand{\R}{\mathbb{R}}
\newcommand{\Rz}{\mathcal{R}}
\newcommand{\QQ}{\mathcal Q}
\newcommand{\VV}{\mathcal V}
\newcommand{\ASM}{\mathrm{ASM}}
\newcommand{\RASM}{\mathrm{RASM}}

\newcommand{\kb}{\tt}
\newcommand{\Pk}[1]{\ensuremath{P_{#1}}}
\newcommand{\Qk}[1]{\ensuremath{Q_{#1}}}
\newcommand{\Pkdisc}[1]{\ensuremath{P_{#1}^{\text{disc}}}}
\newcommand{\Qkdisc}[1]{\ensuremath{Q_{#1}^{\text{disc}}}}
\newcommand{\blue}{\textcolor{blue}}
\newcommand{\green}{\textcolor{green!70!black}}
\newcommand{\red}{\textcolor{red}}
\newcommand{\brown}{\textcolor{brown}}
\newcommand{\cyan}{\textcolor{cyan}}
\newcommand{\magenta}{\textcolor{magenta}}
\newcommand{\yellow}{\textcolor{yellow}}
\newcommand{\mini}{\mathop{\rm minimize}}
\newcommand{\st}{\mbox{subject to }}
\newcommand{\lap}{\Delta}
\newcommand\mtab{\hspace{\stretch{1}}}
\newcommand\ud{\,\mathrm{d}}
\newcommand\bslash{{$\backslash$}}
\newcommand\half{{\frac 1 2}}
\newcommand{\abs}[1]{\left\lvert #1 \right\rvert}
\newcommand{\bigabs}[1]{\big\lvert #1 \big\rvert}
\newcommand{\norm}[1]{\left\lVert #1 \right\rVert}
\newcommand\oneitem[1]{\begin{itemize} \item #1 \end{itemize}}
\newcommand\pfrak{{\mathfrak p}}
\newcommand\nfrak{{\mathfrak n}}
\newcommand\ff{\bm f}
\newcommand\mm{\bm m}
\newcommand\nn{\bm n}
\newcommand\uu{\bm u}
\newcommand\vv{\bm v}
\newcommand\ww{\bm w}
\newcommand\DD{D}
\newcommand{\tcolon}{\!:\!}
\DeclareMathOperator{\sgn}{sgn}
\DeclareMathOperator{\card}{card}
\DeclareMathOperator{\trace}{tr}
\DeclareMathOperator{\erf}{erf}
\DeclareMathOperator{\sspan}{span}
\DeclareMathOperator{\argmin}{arg\,min}
\renewcommand{\bar}{\overline}
% \DeclareMathOperator{\divergence}{div}
% \renewcommand\div\divergence
\renewcommand{\div}{{\nabla \cdot}}
\newcommand\spliceop{\leftrightsquigarrow}
\newcommand\splice[5]{{#1} \overset{#5}{\underset{#3,#4}{\leftrightsquigarrow}} {#2}}
\newcommand{\ip}[2]{{\left\langle #1, #2 \right\rangle}}
\newcommand{\Linfty}{{L^\infty}}

% Dimensionless numbers
\newcommand{\Peclet}{{\mathrm{Pe}}}
\newcommand{\Reynolds}{{\mathrm{Re}}}
\newcommand{\Rayleigh}{{\mathrm{Ra}}}
\newcommand{\Mach}{{\mathrm{Ma}}}
\newcommand{\Prandtl}{{\mathrm{Pr}}}
\newcommand{\Grashof}{{\mathrm{Gr}}}

\newcommand{\sw}[1]{\textsf{\small #1}}
\newcommand{\PETSc}{\sw{PETSc}\xspace}
\newcommand{\PyClaw}{\sw{PyClaw}\xspace}
\newcommand{\Dohp}{\sw{Dohp}\xspace}
\newcommand\libmesh{\sw{libMesh}\xspace}
\newcommand\dealii{\sw{Deal.II}\xspace}
\newcommand\MatMult{\cverb|MatMult|}
\newcommand\MatSolve{\cverb|MatSolve|}
\newcommand{\secref}[1]{{Section~\ref{#1}}}
\newcommand{\chapref}[1]{{Chapter~\ref{#1}}}
\newcommand{\figref}[1]{{Figure~\ref{#1}}}
\newcommand{\tabref}[1]{{Table~\ref{#1}}}
\newcommand\AIJ{{\cverb|AIJ|}}
\newcommand\AIJInode{\cverb|AIJ|/\cverb|Inode|}
\newcommand\BAIJ[1][]{\ifthenelse{\equal{#1}{}}{\cverb|BAIJ|}{\ensuremath{\cverb|BAIJ|(#1)}}}
\newcommand\SBAIJ[1][]{\ifthenelse{\equal{#1}{}}{\cverb|SBAIJ|}{\ensuremath{\cverb|SBAIJ|(#1)}}}
\newcommand\todo[1]{{\color{red}\bf [TODO: #1]}}
\newcommand\tf[1]{\hat{#1}}     % test functions


\title{Tightly coupled solvers with loosely coupled software}
\subtitle{Modular linear algebra for multi-physics}

\author{Jed Brown}


% - Use the \inst command only if there are several affiliations.
% - Keep it simple, no one is interested in your street address.
\institute[ETH Z\"urich]
{
  Laboratory of Hydrology, Hydraulics, and Glaciology \\
  ETH Z\"urich
}

\date{KAUST 2011-03-27}

% This is only inserted into the PDF information catalog. Can be left
% out.
\subject{Talks}


% If you have a file called "university-logo-filename.xxx", where xxx
% is a graphic format that can be processed by latex or pdflatex,
% resp., then you can add a logo as follows:

% \pgfdeclareimage[height=0.5cm]{university-logo}{university-logo-filename}
% \logo{\pgfuseimage{university-logo}}

\AtBeginSection[]
{
\begin{frame}<beamer>
\frametitle{Outline}
\tableofcontents[currentsection]
\end{frame}
}

% Delete this, if you do not want the table of contents to pop up at
% the beginning of each subsection:
% \AtBeginSubsection[]
% {
% \begin{frame}<beamer>
% \frametitle{Outline}
% \tableofcontents[currentsection,currentsubsection]
% \end{frame}
% }

% If you wish to uncover everything in a step-wise fashion, uncomment
% the following command:

%\beamerdefaultoverlayspecification{<+->}

\begin{document}
\lstset{language=C}
\normalem

\begin{frame}
\titlepage
\end{frame}

\begin{frame}
\frametitle{Outline}
\tableofcontents
% You might wish to add the option [pausesections]
\end{frame}

\section{Throughput for matrices}
\input{slides/JFNKBottlenecks.tex}
\input{slides/HardwareCapability.tex}
\begin{frame}[shrink=1]{Memory Bandwidth}
%\todo{Replace with performance numbers for current CPUs.}
\begin{itemize}
\item Stream Triad benchmark (GB/s): $\bm w \gets \alpha \bm x + \bm y$
\includegraphics[width=0.8\textwidth]{figures/StreamTriadXT5VsBGP} \\
\item Sparse matrix-vector product: 6 bytes per flop
\includegraphics[width=0.8\textwidth]{figures/SparseMatVec} \\
% {\footnotesize (from Dinesh Kaushik)}
\item Can test on your system using: {\kb cd \$PETSC\_DIR \&\& make streams}
\end{itemize}
\end{frame}

\input{slides/SpMVPerformanceModel.tex}
\input{slides/OptimizingSpMV.tex}
\input{slides/BAIJPerformance.tex}
\begin{frame}{Optimizing unassembled Mat-Vec}
  \begin{itemize}
  \item High order spatial discretizations do more work per node
    \begin{itemize}
    \item Dense tensor product kernel (like small BLAS3)
    \item Cubic ($Q_3$) elements in 3D can achieve $>70\%$ of peak FPU \\
      (compare to $< 5\%$ for assembled operators on multicore)
    \item Can store Jacobian information at quadrature points \\
      (usually pays off for $Q_2$ and higher in 3D)
    \item Spectral, WENO, DG, FD
    \item Often still need an assembled operator for preconditioning
    \end{itemize}
  \item Boundary element methods
    \begin{itemize}
    \item Dense kernels
    \item Fast Multipole Method (FMM)
    \end{itemize}
  \item<2> \alert{Preconditioning requires more effort}
    \begin{itemize}
    \item Useful to have code to assemble matrices: try out new methods quickly
    \end{itemize}
  \end{itemize}
\end{frame}

\input{slides/Dohp/StokesScaling.tex}
\begin{frame}{What you can do}
  \begin{itemize}
  \item Speak at the most specific language possible
    \begin{itemize}
    \item 3D structural analysis: symmetric block size 3
    \item 3D compressible flow: nonsymmetric block size 5
    \end{itemize}
  \item Order unknowns for cache reuse (low-bandwidth like RCM is good)
  \item Dual order
    \begin{itemize}
    \item Assemble a low-order discretization
    \item Provide matrix-free high-order operator \\
      (FD, ADI, caching at quadrature points)
    \item More robust with SOR and ILU due to $h$-ellipticity
    \item Sometimes Picard linearization has a more compact stencil
    \end{itemize}
  \end{itemize}
\end{frame}

\section{Stiffness}
\input{slides/WhyImplicit.tex}
\input{slides/CoupledMultiphysics.tex}
\input{slides/Anisotropy.tex}
\input{slides/FieldSplit.tex}
\input{slides/SIPreconditioning.tex}
\input{slides/Stokes/WeakFormNewtonStep.tex}
\input{slides/Stokes/PropertiesOfSchurComplement.tex}
\input{slides/Stokes/PreconditioningSchur.tex}
\input{slides/Stokes/NavierStokesSchur.tex}

\section{Coupling}
\input{slides/SNES/FlowControl.tex}
\begin{frame}{Overwhelmed with choices}
  \begin{itemize}
  \item If you have a hard problem, no black-box solver will work well
  \item Everything in PETSc has a plugin architecture
    \begin{itemize}
    \item Put in the ``special sauce'' for your problem
    \item Your implementations are first-class
    \end{itemize}
  \item PETSc exposes an algebra of composition at runtime
    \begin{itemize}
    \item Build a good solver from existing components, at runtime
    \item Multigrid, domain decomposition, factorization, relaxation, field-split
    \item Choose matrix format that works best with your preconditioner
    \item structural blocking, Neumann matrices, monolithic versus nested
    \end{itemize}
  \end{itemize}
\end{frame}
\input{slides/PETSc/Coupling.tex}
\input{slides/PETSc/MatNest.tex}
\input{slides/PETSc/MatGetLocalSubMatrix.tex}

\begin{frame}{Wrap-up}
  \begin{itemize}
  \item Software modularity while retaining access to good solvers
    \begin{itemize}
    \item Reuse single-physics modules
    \item Unintrusive ``special sauce'' (once you figure it out)
    \end{itemize}
  \item Choose the matrix format at runtime, best for your preconditioner
    \begin{itemize}
    \item monolithic, nested, Neumann
    \item scalar or block, symmetric
    \end{itemize}
  \item Break into pieces that are ``understood'', keep some block structure for high throughput
  \end{itemize}
\end{frame}

\end{document}
