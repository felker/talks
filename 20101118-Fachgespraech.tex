%\documentclass[handout]{beamer}
\documentclass{beamer}

\mode<presentation>
{
\usetheme{default}
%\usetheme{Singapore}
%\usetheme{Warsaw}
%\usetheme{Malmoe}
% \useinnertheme{circles}
% \useoutertheme{infolines}
% \useinnertheme{rounded}

\setbeamercovered{transparent}
}

\usepackage[english]{babel}
\usepackage[latin1]{inputenc}
\usepackage{bm,textpos,alltt,listings,multirow,ulem}
\usepackage[amssymb]{SIunits}

% font definitions, try \usepackage{ae} instead of the following
% three lines if you don't like this look
\usepackage{mathptmx}
\usepackage[scaled=.90]{helvet}
\usepackage{courier}
\usepackage[T1]{fontenc}
\usepackage{tikz}
\usetikzlibrary[shapes.arrows,arrows,shapes.misc]

% \usepackage{pgfpages}
% \pgfpagesuselayout{4 on 1}[a4paper,landscape,border shrink=5mm]

\ProvidesPackage{JedMacros}

\usepackage{amsmath}
\usepackage{multirow,listings,booktabs}
\usepackage{mdwlist}
\usepackage{xspace}
\usepackage[iso,american]{isodate}
\makeatletter
\DeclareRobustCommand\onedot{\futurelet\@let@token\@onedot}
\def\@onedot{\ifx\@let@token.\else.\null\fi\xspace}
\def\eg{{e.g}\onedot} \def\Eg{{E.g}\onedot}
\def\ie{{i.e}\onedot} \def\Ie{{I.e}\onedot}
\def\cf{{cf}\onedot} \def\Cf{{Cf}\onedot}
\def\etc{{etc}\onedot}
\def\vs{{vs}\onedot}
\def\wrt{w.r.t\onedot}
\def\dof{d.o.f\onedot}
\def\etal{{et al}\onedot}
\makeatother

\usepackage{tikz}
\usetikzlibrary[shapes,shapes.arrows,arrows,shapes.misc,fit,positioning]

\usepackage{siunitx}
\DeclareSIUnit\year{a}
\DeclareSIUnit\byte{B}
\sisetup{retain-unity-mantissa = false}

\usepackage{fancyvrb}
\usepackage[outputdir=out]{minted}
\newminted{c}{gobble=2}
\newminted{python}{gobble=2}
%\newmint[cverb]{c}{} 
\newcommand\cverb[1][]{\SaveVerb[%
    aftersave={\textnormal{\UseVerb[#1]{vsave}}}]{vsave}}
\newcommand\cfunc[1][]{\SaveVerb[%
    aftersave={\textnormal{\UseVerb[#1]{vsave}\texttt{()}}}]{vsave}}
\newcommand\pyverb[1][]{\SaveVerb[%
    aftersave={\textnormal{\UseVerb[#1]{vsave}}}]{vsave}}
\def\asm#1{{\tt #1}}
\def\code#1{{\tt #1}}
\def\shell#1{{\tt \$ #1}}

\newcommand\email[1]{{\href{mailto:#1}{\nolinkurl{#1}}}}

\newcommand{\II}{\mathcal{I}}
\newcommand{\C}{\mathbb{C}}
\newcommand{\D}{\mathcal{D}}
\newcommand{\EE}{\mathcal{E}}
\newcommand{\F}{\mathcal{F}}
\newcommand{\I}{\mathcal{I}}
\newcommand{\N}{\mathcal{N}}
\newcommand{\PP}{\mathcal{P}}
\newcommand\Ppc{\ensuremath{\mathsf P}}
\newcommand{\bigO}{\ensuremath{\mathcal{O}}}
\newcommand{\R}{\mathbb{R}}
\newcommand{\Rz}{\mathcal{R}}
\newcommand{\QQ}{\mathcal Q}
\newcommand{\VV}{\mathcal V}
\newcommand{\ASM}{\mathrm{ASM}}
\newcommand{\RASM}{\mathrm{RASM}}

\newcommand{\kb}{\tt}
\newcommand{\Pk}[1]{\ensuremath{P_{#1}}}
\newcommand{\Qk}[1]{\ensuremath{Q_{#1}}}
\newcommand{\Pkdisc}[1]{\ensuremath{P_{#1}^{\text{disc}}}}
\newcommand{\Qkdisc}[1]{\ensuremath{Q_{#1}^{\text{disc}}}}
\newcommand{\blue}{\textcolor{blue}}
\newcommand{\green}{\textcolor{green!70!black}}
\newcommand{\red}{\textcolor{red}}
\newcommand{\brown}{\textcolor{brown}}
\newcommand{\cyan}{\textcolor{cyan}}
\newcommand{\magenta}{\textcolor{magenta}}
\newcommand{\yellow}{\textcolor{yellow}}
\newcommand{\mini}{\mathop{\rm minimize}}
\newcommand{\st}{\mbox{subject to }}
\newcommand{\lap}{\Delta}
\newcommand\mtab{\hspace{\stretch{1}}}
\newcommand\ud{\,\mathrm{d}}
\newcommand\bslash{{$\backslash$}}
\newcommand\half{{\frac 1 2}}
\newcommand{\abs}[1]{\left\lvert #1 \right\rvert}
\newcommand{\bigabs}[1]{\big\lvert #1 \big\rvert}
\newcommand{\norm}[1]{\left\lVert #1 \right\rVert}
\newcommand\oneitem[1]{\begin{itemize} \item #1 \end{itemize}}
\newcommand\pfrak{{\mathfrak p}}
\newcommand\nfrak{{\mathfrak n}}
\newcommand\ff{\bm f}
\newcommand\mm{\bm m}
\newcommand\nn{\bm n}
\newcommand\uu{\bm u}
\newcommand\vv{\bm v}
\newcommand\ww{\bm w}
\newcommand\DD{D}
\newcommand{\tcolon}{\!:\!}
\DeclareMathOperator{\sgn}{sgn}
\DeclareMathOperator{\card}{card}
\DeclareMathOperator{\trace}{tr}
\DeclareMathOperator{\erf}{erf}
\DeclareMathOperator{\sspan}{span}
\DeclareMathOperator{\argmin}{arg\,min}
\renewcommand{\bar}{\overline}
% \DeclareMathOperator{\divergence}{div}
% \renewcommand\div\divergence
\renewcommand{\div}{{\nabla \cdot}}
\newcommand\spliceop{\leftrightsquigarrow}
\newcommand\splice[5]{{#1} \overset{#5}{\underset{#3,#4}{\leftrightsquigarrow}} {#2}}
\newcommand{\ip}[2]{{\left\langle #1, #2 \right\rangle}}
\newcommand{\Linfty}{{L^\infty}}

% Dimensionless numbers
\newcommand{\Peclet}{{\mathrm{Pe}}}
\newcommand{\Reynolds}{{\mathrm{Re}}}
\newcommand{\Rayleigh}{{\mathrm{Ra}}}
\newcommand{\Mach}{{\mathrm{Ma}}}
\newcommand{\Prandtl}{{\mathrm{Pr}}}
\newcommand{\Grashof}{{\mathrm{Gr}}}

\newcommand{\sw}[1]{\textsf{\small #1}}
\newcommand{\PETSc}{\sw{PETSc}\xspace}
\newcommand{\PyClaw}{\sw{PyClaw}\xspace}
\newcommand{\Dohp}{\sw{Dohp}\xspace}
\newcommand\libmesh{\sw{libMesh}\xspace}
\newcommand\dealii{\sw{Deal.II}\xspace}
\newcommand\MatMult{\cverb|MatMult|}
\newcommand\MatSolve{\cverb|MatSolve|}
\newcommand{\secref}[1]{{Section~\ref{#1}}}
\newcommand{\chapref}[1]{{Chapter~\ref{#1}}}
\newcommand{\figref}[1]{{Figure~\ref{#1}}}
\newcommand{\tabref}[1]{{Table~\ref{#1}}}
\newcommand\AIJ{{\cverb|AIJ|}}
\newcommand\AIJInode{\cverb|AIJ|/\cverb|Inode|}
\newcommand\BAIJ[1][]{\ifthenelse{\equal{#1}{}}{\cverb|BAIJ|}{\ensuremath{\cverb|BAIJ|(#1)}}}
\newcommand\SBAIJ[1][]{\ifthenelse{\equal{#1}{}}{\cverb|SBAIJ|}{\ensuremath{\cverb|SBAIJ|(#1)}}}
\newcommand\todo[1]{{\color{red}\bf [TODO: #1]}}
\newcommand\tf[1]{\hat{#1}}     % test functions


\title{Computing free surface flows}

\author{Jed Brown}


% - Use the \inst command only if there are several affiliations.
% - Keep it simple, no one is interested in your street address.
\institute[ETH Z\"urich]
{
  Laboratory of Hydrology, Hydraulics, and Glaciology \\
  ETH Z\"urich
}

\date[2010-11-18]{Fachgespr\"ach 2010-11-18}

% This is only inserted into the PDF information catalog. Can be left
% out.
\subject{Talks}


% If you have a file called "university-logo-filename.xxx", where xxx
% is a graphic format that can be processed by latex or pdflatex,
% resp., then you can add a logo as follows:

% \pgfdeclareimage[height=0.5cm]{university-logo}{university-logo-filename}
% \logo{\pgfuseimage{university-logo}}



% Delete this, if you do not want the table of contents to pop up at
% the beginning of each subsection:
% \AtBeginSubsection[]
% {
% \begin{frame}<beamer>
% \frametitle{Outline}
% \tableofcontents[currentsection,currentsubsection]
% \end{frame}
% }

% If you wish to uncover everything in a step-wise fashion, uncomment
% the following command:

%\beamerdefaultoverlayspecification{<+->}

\begin{document}
\lstset{language=C}
\normalem

\begin{frame}
\titlepage
\end{frame}

\begin{frame}
  \frametitle{Outline}
  \tableofcontents
  % You might wish to add the option [pausesections]
\end{frame}

\section{Motivation}
\input{slides/GroundingLine/BindschadlerNaturalHistory.tex}
\input{slides/GroundingLine/SchoofRetreat.tex}
\input{slides/GroundingLine/Sensitivity.tex}
\input{slides/GroundingLine/MeshDependence.tex}
\input{slides/GroundingLine/ThwaitesPIG.tex}
\input{slides/GroundingLine/YPlus.tex}

\section{ALE Formulation}
\input{slides/GroundingLine/ALEFormulation.tex}
%\input{slides/Dohp/Resolution.tex}
\input{slides/GroundingLine/ALEBlockForm.tex}
\input{slides/Dohp/StokesScaling.tex}

\section{Conservation}
\input{slides/InfSup/Stabilized.tex}

\section[Slip]{Slip boundary conditions on bumpy surfaces}
\input{slides/GroundingLine/Normals.tex}

\section[Hydrostatic]{A robust multigrid solver for the hydrostatic equations}
\begin{frame}[shrink=5]{Hydrostatic equations for ice sheet flow}
  \begin{itemize}
  \item Valid when $w_x \ll u_z$, independent of basal friction {\small (Schoof\&Hindmarsh 2010)}
  \item Eliminate $p$ and $w$ from Stokes by incompressibility:\\
    \quad 3D elliptic system for $\bm u = (u,v)$
    \begin{align*}
      - \nabla\cdot \left[ \eta
        \begin{pmatrix}
          4 u_x + 2 v_y & u_y + v_x & u_z \\
          u_y + v_x & 2 u_x + 4 v_y & v_z
        \end{pmatrix} \right] + \rho g \bar\nabla h & = 0
    \end{align*}
    \begin{align*}
      \eta(\theta,\gamma) &= \frac{B(\theta)}{2} (\gamma_0 + \gamma)^{\frac{1-n}{2n}}, \qquad n \approx 3 \\
      \gamma &= u_x^2 + v_y^2 + u_xv_y + \frac 1 4 (u_y+v_x)^2 + \frac 1 4 u_z^2 + \frac 1 4 v_z^2
    \end{align*}
    and slip boundary $\sigma \cdot \bm n = \beta^2 \bm u$ where
    \begin{align*}
      \beta^2(\gamma_b) &= \beta_0^2 (\epsilon_b^2 + \gamma_b)^{\frac{m-1}{2}}, \qquad 0 < m \le 1 \\
      \gamma_b &= \frac 1 2 (u^2 + v^2)
    \end{align*}
  \item $Q_1$ FEM with Newton-Krylov-Multigrid solver in PETSc: \code{src/snes/examples/tutorials/ex48.c}
  \end{itemize}
\end{frame}

\input{slides/THI/X5kmClip.tex}
\input{slides/THI/Y5kmClip.tex}
\input{slides/THI/WhatAboutSplitting.tex}

\begin{frame}{Outlook}
  \begin{block}{}
    \begin{itemize}
    \item Exact local conservation is critical for problems with discontinuous geometry and coefficients
    \item Nonlinear slip on irregular surfaces is hard but tractable (mostly)
    \item Smooth manufactured solutions are necessary, but not sufficient to study solver and discretization performance
    \item Need good software to combine relaxation for loosely coupled processes and factorization
      for stiff/indefinite coupling
    \item Modeling of boundary layer processes in highly anisotropic geometry likely requires conforming to the interface
    \end{itemize}    
  \end{block}
  \begin{block}{Tools}
    \begin{itemize}
    \item PETSc\ \url{http://mcs.anl.gov/petsc}
      \begin{itemize}\item ML, Hypre, MUMPS
      \end{itemize}
    \item ITAPS \url{http://itaps.org}
      \begin{itemize}\item MOAB, CGM, Lasso
      \end{itemize}
    \end{itemize}
  \end{block}
\end{frame}
\end{document}
