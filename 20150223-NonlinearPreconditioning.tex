% \documentclass[handout]{beamer}
\documentclass{beamer}

\mode<presentation>
{
  \usetheme{ANLBlue}
  % \usefonttheme[onlymath]{serif}
  % \usetheme{Singapore}
  % \usetheme{Warsaw}
  % \usetheme{Malmoe}
  % \useinnertheme{circles}
  % \useoutertheme{infolines}
  % \useinnertheme{rounded}

  \setbeamercovered{transparent=1}
}

\usepackage[english]{babel}
\usepackage[latin1]{inputenc}
\usepackage{alltt,listings,multirow,ulem,siunitx}
\usepackage[absolute,overlay]{textpos}
\TPGrid{1}{1}
\usepackage{pdfpages}
\usepackage{ulem}
\usepackage{multimedia}
\usepackage{multicol}
\newcommand\hmmax{0}
\newcommand\bmmax{0}
\usepackage{bm}
\usepackage{comment}
\usepackage{subcaption}

% font definitions, try \usepackage{ae} instead of the following
% three lines if you don't like this look
\usepackage{mathptmx}
\usepackage[scaled=.90]{helvet}
% \usepackage{courier}
\usepackage[T1]{fontenc}
\usepackage{tikz}
\usetikzlibrary{decorations.pathreplacing}
\usetikzlibrary{shadows,arrows,shapes.misc,shapes.arrows,shapes.multipart,arrows,decorations.pathmorphing,backgrounds,positioning,fit,petri,calc,shadows,chains,matrix}

\newcommand\vvec{\bm v}
\newcommand\bvec{\bm b}
\newcommand\bxk{\bvec_0 \times \kappa_0 \cdot \nabla}
\newcommand\delp{\nabla_\perp}

% \usepackage{pgfpages}
% \pgfpagesuselayout{4 on 1}[a4paper,landscape,border shrink=5mm]

\usepackage{JedMacros}

\newcommand{\timeR}{t_{\mathrm{R}}}
\newcommand{\timeW}{t_{\mathrm{W}}}
\newcommand{\mglevel}{\ensuremath{\ell}}
\newcommand{\mglevelcp}{\ensuremath{\mglevel_{\mathrm{cp}}}}
\newcommand{\mglevelcoarse}{\ensuremath{\mglevel_{\mathrm{coarse}}}}
\newcommand{\mglevelfine}{\ensuremath{\mglevel_{\mathrm{fine}}}}

%solution and residual
\newcommand{\vx}{\ensuremath{x}}
\newcommand{\vc}{\ensuremath{\hat{x}}}
\newcommand{\vr}{\ensuremath{r}}
\newcommand{\vb}{\ensuremath{b}}

%operators
\newcommand{\vA}{\ensuremath{A}}
\newcommand{\vP}{\ensuremath{I_H^h}}
\newcommand{\vS}{\ensuremath{S}}
\newcommand{\vR}{\ensuremath{I_h^H}}
\newcommand{\vI}{\ensuremath{\hat I_h^H}}
\newcommand{\vV}{\ensuremath{\mathbf{V}}}
\newcommand{\vF}{\ensuremath{F}}
\newcommand{\vtau}{\ensuremath{\mathbf{\tau}}}


\title{Nonlinear Preconditioning and Time Integration}
\author{{\bf Jed Brown} \texttt{jed@jedbrown.org} \\
}

% - Use the \inst command only if there are several affiliations.
% - Keep it simple, no one is interested in your street address.
% \institute
% {
%   Mathematics and Computer Science Division \\ Argonne National Laboratory
% }

\date{CU Boulder, 2015-02-23}
% \\[1em]
%This talk: \url{http://59A2.org/files/20150213-AdaptHeterogeneous.pdf}}

% This is only inserted into the PDF information catalog. Can be left
% out.
\subject{Talks}


% If you have a file called "university-logo-filename.xxx", where xxx
% is a graphic format that can be processed by latex or pdflatex,
% resp., then you can add a logo as follows:

% \pgfdeclareimage[height=0.5cm]{university-logo}{university-logo-filename}
% \logo{\pgfuseimage{university-logo}}



% Delete this, if you do not want the table of contents to pop up at
% the beginning of each subsection:
% \AtBeginSubsection[]
% {
% \begin{frame}<beamer>
%   \frametitle{Outline}
%   \tableofcontents[currentsection,currentsubsection]
% \end{frame}
% }

% \AtBeginSection[]
% {
%   \begin{frame}<beamer>
%     \frametitle{Outline}
%     \tableofcontents[currentsection]
%   \end{frame}
% }

% If you wish to uncover everything in a step-wise fashion, uncomment
% the following command:

% \beamerdefaultoverlayspecification{<+->}

\begin{document}
\lstset{language=C}
\normalem

\begin{frame}
  \titlepage
\end{frame}

\begin{frame}{Nonlinear solvers}
  \begin{gather*}
    F(x) = b \\
    r(x) = F(x) - b
  \end{gather*}
  \begin{block}{Solution techniques}
    \begin{itemize}
    \item Newton-Krylov
      \begin{itemize}
      \item global linearization
      \item local nonlinearities can slow global convergence
      \end{itemize}
    \item Quasi-Newton (build low-rank corrections to approximate Jacobian inverse)
      \begin{itemize}
      \item Amortize setup for linear solver
      \end{itemize}
    \item Nonlinear multigrid (FAS)
    \item Nonlinear accelerators (NGMRES)
    \item Methods can be \emph{composed}
    \end{itemize}
  \end{block}
\end{frame}

\begin{frame}{Left preconditioning}
  \begin{itemize}
  \item Approximate solver $x_{i+1} = M(F,x_i,b)$, similar to $x_i - P^{-1}(A x_i - b)$
  \item Look for a fixed point: $r(x) = x - M(F,x,b)$
  \item Analogous to linear case
    \begin{equation*}
      r(x) = P^{-1} (A x - b) = x + [P^{-1} (A x - b) - x] = x - M(A,x,b)
    \end{equation*}
  \item Important feature: line search in preconditioned norm
  \end{itemize}
\end{frame}

\begin{frame}{Right Preconditioning}
  \begin{itemize}
  \item Linear analogy: $A P^{-1} \underbrace{(P x)}_y = b$
  \item Nonlinear version: $F(M(F,y,b)) = b$
    \begin{itemize}
    \item After finding solution $y^*$, get $x = M(F,y^*,b)$
    \end{itemize}
  \item Important: line search in unpreconditioned norm
  \end{itemize}
\end{frame}

\input{slides/MG/TauFAS.tex}

\begin{frame}{IMEX time integration in PETSc}
  \begin{itemize}
  \item Additive Runge-Kutta IMEX methods
    \begin{gather*}
      G(t,x,\dot x) = F(t,x) \\
      J_\alpha = \alpha G_{\dot x} + G_x
    \end{gather*}
    \vspace{-1em}
    \begin{itemize}
    \item User provides:
      \begin{itemize}
      \item \texttt{FormRHSFunction(ts,$t$,$x$,$F$,void *ctx);}
      \item \texttt{FormIFunction(ts,$t$,$x$,$\dot x$,$G$,void *ctx);}
      \item \texttt{FormIJacobian(ts,$t$,$x$,$\dot x$,$\alpha$,$J$,$J_{p}$,mstr,void *ctx);}
      \end{itemize}
    \end{itemize}
  \item Stiff terms in $G$ to be solved implicitly
  \item Non-stiff terms in $F$
    \begin{itemize}
    \item Important for non-smooth terms, e.g., flux/slope limiters
    \item Fewer evaluations because no iterative solve
    \end{itemize}
  \item Rosenbrock-W: amortize assembly and setup
  \end{itemize}
\end{frame}

\begin{frame}{Experiments}
  \begin{itemize}
  \item \url{http://www.mcs.anl.gov/publication/composing-scalable-nonlinear-algebraic-solvers}
  \item \url{http://www.mcs.anl.gov/publication/low-rank-quasi-newton-updates-robust-jacobian-lagging-newton-methods}
  \item Nonlinear elasticity \texttt{src/snes/examples/tutorials/ex16.c}
  \item Thermo-lid-driven cavity \texttt{src/snes/examples/tutorials/ex19.c}
  \item Transient version: \texttt{src/ts/examples/tutorials/ex26.c}
  \item Ice sheet flow \texttt{src/snes/examples/tutorials/ex48.c}
  \item Compare cost
    \begin{itemize}
    \item \texttt{SNESComputeJacobian}
    \item \texttt{SNESComputeFunction}
    \item \texttt{PCSetUp}
    \item \texttt{PCApply}
    \item \texttt{SNESSolve}
    \end{itemize}
  \end{itemize}
\end{frame}

\end{document}
