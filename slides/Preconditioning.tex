%\subsection{Preconditioning}
\begin{frame}{Preconditioning}
  \begin{block}{Idea: improve the conditioning of the Krylov operator}
    \begin{itemize}
    \item Left preconditioning
      \vspace{-1em}
      \begin{gather*}
        (P^{-1} A) x = P^{-1} b \\
        \{ P^{-1} b, (P^{-1}A) P^{-1} b, (P^{-1}A)^2 P^{-1} b, \dotsc \}
      \end{gather*}
    \item Right preconditioning
      \vspace{-1em}
      \begin{gather*}
        (A P^{-1}) P x = b \\
        \{ b, (A P^{-1}b, (A P^{-1})^2b, \dotsc \}
      \end{gather*}
    \item The product $P^{-1}A$ or $A P^{-1}$ is \emph{not} formed.
    \end{itemize}
  \end{block}
  \begin{definition}[Preconditioner]
      A \emph{preconditioner} $\PP$ is a method for constructing a
matrix (just a linear function, not assembled!)  $P^{-1} = \PP(A,A_p)$
using a matrix $A$ and extra information $A_p$, such that the spectrum
of $P^{-1}A$ (or $A P^{-1}$) is well-behaved.
    \end{definition}
\end{frame}

\input{slides/PreconditionerDefinition.tex}

\begin{frame}{Questions to ask when you see a matrix}
  \begin{enumerate}
  \item What do you want to do with it?
    \begin{itemize}
    \item Multiply with a vector
    \item Solve linear systems or eigen-problems
    \end{itemize}
  \item How is the conditioning/spectrum?
    \begin{itemize}
    \item distinct/clustered eigen/singular values?
    \item symmetric positive definite ($\sigma(A) \subset \R^+$)?
    \item nonsymmetric definite ($\sigma(A) \subset \{z \in \C : \Re [z] > 0 \}$)?
    \item indefinite?
    \end{itemize}
  \item How dense is it?
    \begin{itemize}
    \item block/banded diagonal?
    \item sparse unstructured?
    \item denser than we'd like?
    \end{itemize}
  \item Is there a better way to compute $Ax$?
  \item Is there a different matrix with similar spectrum, but nicer properties?
  \item \alert<2>{How can we precondition $A$?}
  \end{enumerate}
\end{frame}

\begin{frame}{Relaxation}
  Split into lower, diagonal, upper parts: \alert{$ A = L + D + U $}
  \begin{block}{Jacobi}
    Cheapest preconditioner: $P^{-1} = D^{-1}$
  \end{block}
  \begin{block}{Successive over-relaxation (SOR)}
    \begin{gather*}
      \left(L + \frac 1 \omega D\right) x_{n+1} = \left[\left(\frac
          1\omega-1\right)D - U\right] x_n + \omega b \\
      P^{-1} = \text{$k$ iterations starting with $x_0=0$}
    \end{gather*}
    \begin{itemize}
    \item Implemented as a sweep
    \item $\omega = 1$ corresponds to Gauss-Seidel
    \item Very effective at removing high-frequency components of residual
    \end{itemize}
  \end{block}
\end{frame}

\begin{frame}[shrink=5]{Factorization}
  Two phases
  \begin{itemize}
  \item symbolic factorization: find where fill occurs, only uses sparsity pattern
  \item numeric factorization: compute factors
  \end{itemize}
  \begin{block}{LU decomposition}
    \begin{itemize}
    \item Ultimate preconditioner
    \item Expensive, for $m\times m$ sparse matrix with bandwidth $b$, traditionally requires $\bigO(mb^2)$ time and $\bigO(mb)$ space.
      \begin{itemize}
      \item Bandwidth scales as $m^{\frac{d-1}{d}}$ in $d$-dimensions
      \item Optimal in 2D: $\bigO(m \cdot \log m)$ space, $\bigO(m^{3/2})$ time
      \item Optimal in 3D: $\bigO(m^{4/3})$ space, $\bigO(m^2)$ time
      \end{itemize}
    \item Symbolic factorization is problematic in parallel
    \end{itemize}
  \end{block}
  \begin{block}{Incomplete LU}
    \begin{itemize}
    \item Allow a limited number of levels of fill:
      ILU($k$)
    \item Only allow fill for entries that exceed threshold: ILUT
    \item Usually poor scaling in parallel
    \item No guarantees
    \end{itemize}
  \end{block}
\end{frame}

\begin{frame}{1-level Domain decomposition}
  Domain size $L$, subdomain size $H$, element size $h$
  \begin{block}{Overlapping/Schwarz}
    \begin{itemize}\item Solve Dirichlet problems on overlapping
      subdomains
    \item No overlap: $\textit{its} \in \bigO\big( \frac{L}{\sqrt{Hh}} \big)$
    \item Overlap $\delta$: $\textit{its} \in \big( \frac L {\sqrt{H\delta}} \big)$
    \end{itemize}
  \end{block}
  \vspace{-1ex}
  \begin{block}{Neumann-Neumann}
    \begin{itemize}
    \item Solve Neumann problems on non-overlapping subdomains
    \item $\textit{its} \in \bigO\big( \frac{L}{H}(1+\log\frac H h) \big)$
    \item Tricky null space issues (floating subdomains)
    \item Need subdomain matrices, not globally assembled matrix.
    \end{itemize}
  \end{block}
  \begin{itemize}
  \item Multilevel variants knock off the leading $\frac L H$
  \item Both overlapping and nonoverlapping with this bound
  \end{itemize}
  % \begin{block}{BDDC and FETI-DP}
  %   \begin{itemize}
  %   \item Neumann problems on subdomains with
  %     coarse grid correction
  %   \item $\textit{its} \in \bigO\big(1 + \log\frac H h \big)$
  %   \end{itemize}
  %   \includegraphics[width=0.7\textwidth]{bddc}
  % \end{block}
\end{frame}

\begin{frame}[shrink=5]{Multigrid}
  \begin{block}{Hierarchy: Interpolation and restriction operators}
    \begin{equation*}
    \II^\uparrow : X_{\text{coarse}} \to X_{\text{fine}} \qquad
    \II^\downarrow :  X_{\text{fine}} \to X_{\text{coarse}}
  \end{equation*}
  \end{block}
  \begin{itemize}
  \item Geometric: define problem on multiple levels, use grid to compute hierarchy
  \item Algebraic: define problem only on finest level, use matrix structure to build hierarchy
  \end{itemize}
  \begin{block}{Galerkin approximation}
    Assemble this matrix: $A_{\text{coarse}} = \II^\downarrow A_{\text{fine}} \II^\uparrow$
  \end{block}
  \begin{block}{Application of multigrid preconditioner ($V$-cycle)}
    \begin{itemize}
    \item Apply pre-smoother on fine level (any preconditioner)
    \item Restrict residual to coarse level with $\II^\downarrow$
    \item Solve on coarse level $A_{\text{coarse}} x = r$
    \item Interpolate result back to fine level with $\II^\uparrow$
    \item Apply post-smoother on fine level (any preconditioner)
    \end{itemize}
  \end{block}
\end{frame}

\begin{frame}{Multigrid convergence properties}
  \begin{itemize}
  \item Textbook: $P^{-1}A$ is spectrally equivalent to identity
    \begin{itemize}
    \item Constant number of iterations to converge up to discretization error
    \end{itemize}
  \item Most theory applies to SPD systems
    \begin{itemize}
    \item variable coefficients (\eg discontinuous): low energy interpolants
    \item mesh- and/or physics-induced anisotropy: semi-coarsening/line smoothers
    \item complex geometry: difficult to have meaningful coarse levels
    \end{itemize}
  \item Deeper algorithmic difficulties
    \begin{itemize}
    \item nonsymmetric (\eg advection, shallow water, Euler) \\
    \item indefinite (\eg incompressible flow, Helmholtz)
    \end{itemize}
  \item Performance considerations
    \begin{itemize}
    \item Aggressive coarsening is critical in parallel
    \item Most theory uses SOR smoothers, ILU often more robust
    \item Coarsest level usually solved semi-redundantly with direct solver
    \end{itemize}
  \item Multilevel Schwarz is essentially the same with different language
    \begin{itemize}
    \item assume strong smoothers, emphasize aggressive coarsening
    \end{itemize}
  \end{itemize}
\end{frame}
