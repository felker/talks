%\documentclass[handout]{beamer}
\documentclass{beamer}

\mode<presentation>
{
\usetheme{default}
\usefonttheme[onlymath]{serif}
%\usetheme{Singapore}
%\usetheme{Warsaw}
%\usetheme{Malmoe}
% \useinnertheme{circles}
% \useoutertheme{infolines}
% \useinnertheme{rounded}

\setbeamercovered{transparent=5}
}

\usepackage[english]{babel}
\usepackage[latin1]{inputenc}
\usepackage{textpos,alltt,listings,multirow,ulem,siunitx}
\newcommand\hmmax{0}
\newcommand\bmmax{0}
\usepackage{bm}

% font definitions, try \usepackage{ae} instead of the following
% three lines if you don't like this look
\usepackage{mathptmx}
\usepackage[scaled=.90]{helvet}
%\usepackage{courier}
\usepackage[T1]{fontenc}
\usepackage{tikz}
\usetikzlibrary[shapes,shapes.arrows,arrows,shapes.misc,fit,positioning]

% \usepackage{pgfpages}
% \pgfpagesuselayout{4 on 1}[a4paper,landscape,border shrink=5mm]

\ProvidesPackage{JedMacros}

\usepackage{amsmath}
\usepackage{multirow,listings,booktabs}
\usepackage{mdwlist}
\usepackage{xspace}
\usepackage[iso,american]{isodate}
\makeatletter
\DeclareRobustCommand\onedot{\futurelet\@let@token\@onedot}
\def\@onedot{\ifx\@let@token.\else.\null\fi\xspace}
\def\eg{{e.g}\onedot} \def\Eg{{E.g}\onedot}
\def\ie{{i.e}\onedot} \def\Ie{{I.e}\onedot}
\def\cf{{cf}\onedot} \def\Cf{{Cf}\onedot}
\def\etc{{etc}\onedot}
\def\vs{{vs}\onedot}
\def\wrt{w.r.t\onedot}
\def\dof{d.o.f\onedot}
\def\etal{{et al}\onedot}
\makeatother

\usepackage{tikz}
\usetikzlibrary[shapes,shapes.arrows,arrows,shapes.misc,fit,positioning]

\usepackage{siunitx}
\DeclareSIUnit\year{a}
\DeclareSIUnit\byte{B}
\sisetup{retain-unity-mantissa = false}

\usepackage{fancyvrb}
\usepackage[outputdir=out]{minted}
\newminted{c}{gobble=2}
\newminted{python}{gobble=2}
%\newmint[cverb]{c}{} 
\newcommand\cverb[1][]{\SaveVerb[%
    aftersave={\textnormal{\UseVerb[#1]{vsave}}}]{vsave}}
\newcommand\cfunc[1][]{\SaveVerb[%
    aftersave={\textnormal{\UseVerb[#1]{vsave}\texttt{()}}}]{vsave}}
\newcommand\pyverb[1][]{\SaveVerb[%
    aftersave={\textnormal{\UseVerb[#1]{vsave}}}]{vsave}}
\def\asm#1{{\tt #1}}
\def\code#1{{\tt #1}}
\def\shell#1{{\tt \$ #1}}

\newcommand\email[1]{{\href{mailto:#1}{\nolinkurl{#1}}}}

\newcommand{\II}{\mathcal{I}}
\newcommand{\C}{\mathbb{C}}
\newcommand{\D}{\mathcal{D}}
\newcommand{\EE}{\mathcal{E}}
\newcommand{\F}{\mathcal{F}}
\newcommand{\I}{\mathcal{I}}
\newcommand{\N}{\mathcal{N}}
\newcommand{\PP}{\mathcal{P}}
\newcommand\Ppc{\ensuremath{\mathsf P}}
\newcommand{\bigO}{\ensuremath{\mathcal{O}}}
\newcommand{\R}{\mathbb{R}}
\newcommand{\Rz}{\mathcal{R}}
\newcommand{\QQ}{\mathcal Q}
\newcommand{\VV}{\mathcal V}
\newcommand{\ASM}{\mathrm{ASM}}
\newcommand{\RASM}{\mathrm{RASM}}

\newcommand{\kb}{\tt}
\newcommand{\Pk}[1]{\ensuremath{P_{#1}}}
\newcommand{\Qk}[1]{\ensuremath{Q_{#1}}}
\newcommand{\Pkdisc}[1]{\ensuremath{P_{#1}^{\text{disc}}}}
\newcommand{\Qkdisc}[1]{\ensuremath{Q_{#1}^{\text{disc}}}}
\newcommand{\blue}{\textcolor{blue}}
\newcommand{\green}{\textcolor{green!70!black}}
\newcommand{\red}{\textcolor{red}}
\newcommand{\brown}{\textcolor{brown}}
\newcommand{\cyan}{\textcolor{cyan}}
\newcommand{\magenta}{\textcolor{magenta}}
\newcommand{\yellow}{\textcolor{yellow}}
\newcommand{\mini}{\mathop{\rm minimize}}
\newcommand{\st}{\mbox{subject to }}
\newcommand{\lap}{\Delta}
\newcommand\mtab{\hspace{\stretch{1}}}
\newcommand\ud{\,\mathrm{d}}
\newcommand\bslash{{$\backslash$}}
\newcommand\half{{\frac 1 2}}
\newcommand{\abs}[1]{\left\lvert #1 \right\rvert}
\newcommand{\bigabs}[1]{\big\lvert #1 \big\rvert}
\newcommand{\norm}[1]{\left\lVert #1 \right\rVert}
\newcommand\oneitem[1]{\begin{itemize} \item #1 \end{itemize}}
\newcommand\pfrak{{\mathfrak p}}
\newcommand\nfrak{{\mathfrak n}}
\newcommand\ff{\bm f}
\newcommand\mm{\bm m}
\newcommand\nn{\bm n}
\newcommand\uu{\bm u}
\newcommand\vv{\bm v}
\newcommand\ww{\bm w}
\newcommand\DD{D}
\newcommand{\tcolon}{\!:\!}
\DeclareMathOperator{\sgn}{sgn}
\DeclareMathOperator{\card}{card}
\DeclareMathOperator{\trace}{tr}
\DeclareMathOperator{\erf}{erf}
\DeclareMathOperator{\sspan}{span}
\DeclareMathOperator{\argmin}{arg\,min}
\renewcommand{\bar}{\overline}
% \DeclareMathOperator{\divergence}{div}
% \renewcommand\div\divergence
\renewcommand{\div}{{\nabla \cdot}}
\newcommand\spliceop{\leftrightsquigarrow}
\newcommand\splice[5]{{#1} \overset{#5}{\underset{#3,#4}{\leftrightsquigarrow}} {#2}}
\newcommand{\ip}[2]{{\left\langle #1, #2 \right\rangle}}
\newcommand{\Linfty}{{L^\infty}}

% Dimensionless numbers
\newcommand{\Peclet}{{\mathrm{Pe}}}
\newcommand{\Reynolds}{{\mathrm{Re}}}
\newcommand{\Rayleigh}{{\mathrm{Ra}}}
\newcommand{\Mach}{{\mathrm{Ma}}}
\newcommand{\Prandtl}{{\mathrm{Pr}}}
\newcommand{\Grashof}{{\mathrm{Gr}}}

\newcommand{\sw}[1]{\textsf{\small #1}}
\newcommand{\PETSc}{\sw{PETSc}\xspace}
\newcommand{\PyClaw}{\sw{PyClaw}\xspace}
\newcommand{\Dohp}{\sw{Dohp}\xspace}
\newcommand\libmesh{\sw{libMesh}\xspace}
\newcommand\dealii{\sw{Deal.II}\xspace}
\newcommand\MatMult{\cverb|MatMult|}
\newcommand\MatSolve{\cverb|MatSolve|}
\newcommand{\secref}[1]{{Section~\ref{#1}}}
\newcommand{\chapref}[1]{{Chapter~\ref{#1}}}
\newcommand{\figref}[1]{{Figure~\ref{#1}}}
\newcommand{\tabref}[1]{{Table~\ref{#1}}}
\newcommand\AIJ{{\cverb|AIJ|}}
\newcommand\AIJInode{\cverb|AIJ|/\cverb|Inode|}
\newcommand\BAIJ[1][]{\ifthenelse{\equal{#1}{}}{\cverb|BAIJ|}{\ensuremath{\cverb|BAIJ|(#1)}}}
\newcommand\SBAIJ[1][]{\ifthenelse{\equal{#1}{}}{\cverb|SBAIJ|}{\ensuremath{\cverb|SBAIJ|(#1)}}}
\newcommand\todo[1]{{\color{red}\bf [TODO: #1]}}
\newcommand\tf[1]{\hat{#1}}     % test functions


\title{Implicit solution of free surface flows in glaciology}

\author{Jed Brown}


% - Use the \inst command only if there are several affiliations.
% - Keep it simple, no one is interested in your street address.
\institute[ETH Z\"urich]
{
  Laboratory of Hydrology, Hydraulics, and Glaciology \\
  ETH Z\"urich
}

\date[2011-03-01]{2011-03-01}

% This is only inserted into the PDF information catalog. Can be left
% out.
\subject{Talks}


% If you have a file called "university-logo-filename.xxx", where xxx
% is a graphic format that can be processed by latex or pdflatex,
% resp., then you can add a logo as follows:

% \pgfdeclareimage[height=0.5cm]{university-logo}{university-logo-filename}
% \logo{\pgfuseimage{university-logo}}



% Delete this, if you do not want the table of contents to pop up at
% the beginning of each subsection:
% \AtBeginSubsection[]
% {
% \begin{frame}<beamer>
% \frametitle{Outline}
% \tableofcontents[currentsection,currentsubsection]
% \end{frame}
% }

% If you wish to uncover everything in a step-wise fashion, uncomment
% the following command:

%\beamerdefaultoverlayspecification{<+->}

\begin{document}
\lstset{language=C}
\normalem

\begin{frame}
\titlepage
\end{frame}

% \begin{frame}
%   \frametitle{Outline}
%   \tableofcontents
%   % You might wish to add the option [pausesections]
% \end{frame}

% \section{Practicalities}

\input{slides/GroundingLine/BindschadlerNaturalHistory.tex}

\section[Hydrostatic]{A robust multigrid solver for the hydrostatic equations}
\begin{frame}[shrink=5]{Hydrostatic equations for ice sheet flow}
  \begin{itemize}
  \item Valid when $w_x \ll u_z$, independent of basal friction {\small (Schoof\&Hindmarsh 2010)}
  \item Eliminate $p$ and $w$ from Stokes by incompressibility:\\
    \quad 3D elliptic system for $\bm u = (u,v)$
    \begin{align*}
      - \nabla\cdot \left[ \eta
        \begin{pmatrix}
          4 u_x + 2 v_y & u_y + v_x & u_z \\
          u_y + v_x & 2 u_x + 4 v_y & v_z
        \end{pmatrix} \right] + \rho g \bar\nabla h & = 0
    \end{align*}
    \begin{align*}
      \eta(\theta,\gamma) &= \frac{B(\theta)}{2} (\gamma_0 + \gamma)^{\frac{1-n}{2n}}, \qquad n \approx 3 \\
      \gamma &= u_x^2 + v_y^2 + u_xv_y + \frac 1 4 (u_y+v_x)^2 + \frac 1 4 u_z^2 + \frac 1 4 v_z^2
    \end{align*}
    and slip boundary $\sigma \cdot \bm n = \beta^2 \bm u$ where
    \begin{align*}
      \beta^2(\gamma_b) &= \beta_0^2 (\epsilon_b^2 + \gamma_b)^{\frac{m-1}{2}}, \qquad 0 < m \le 1 \\
      \gamma_b &= \frac 1 2 (u^2 + v^2)
    \end{align*}
  \item $Q_1$ FEM with Newton-Krylov-Multigrid solver in PETSc: \code{src/snes/examples/tutorials/ex48.c}
  \end{itemize}
\end{frame}

\begin{frame}
  \includegraphics[width=\textwidth]{figures/THI/x-shear}
\end{frame}
\begin{frame}
  \includegraphics[width=\textwidth]{figures/THI/x-80km-m16p2l6-ew} \\
  Grid-sequenced Newton-Krylov solution of test $X$.  The solid lines denote nonlinear iterations, and the dotted lines with $\times$ denote linear residuals.
\end{frame}
\input{slides/THI/Y5kmClip.tex}
\begin{frame}
  \begin{figure}
    \includegraphics[width=\textwidth]{figures/THI/y-10km-m10p6l5-ew}
    \centering\caption{Grid sequenced Newton-Krylov convergence for test $Y$.
    The ``cliff'' has \SI{58}{\degree} angle in the red line ($12\times 125$ meter elements), \SI{73}{\degree} for the cyan line ($6\times 62$ meter elements).}\label{fig:testy}
  \end{figure}
\end{frame}
\begin{frame}
  \begin{figure}
    \includegraphics[width=\textwidth]{figures/THI/linear4}
    \centering\caption{Average number of Krylov iterations per nonlinear iteration.  Each nonlinear system was solved to a relative tolerance of $10^{-2}$.}\label{fig:linear}
  \end{figure}
\end{frame}

\input{slides/THI/Shaheen.tex}
\begin{frame}
  \includegraphics[width=0.8\textwidth]{figures/THI/c-steady-crop}
\end{frame}

% \section{Other models}
% \input{slides/GroundingLine/SSA.tex}
% \begin{frame}{Non-Newtonian Stokes system: velocity $\bm u$, pressure $p$}
% \begin{columns}
% \begin{column}{0.5\textwidth}
%   \alert{\begin{align*}
%     -\nabla \cdot(\eta D\uu) + \nabla p - \ff &= 0 \\
%     \nabla \cdot \uu &= 0
%   \end{align*}}
% \end{column}
% \begin{column}{0.5\textwidth}
%     \begin{align*}
%       D\uu &= \tfrac 1 2 \left(\nabla \uu + (\nabla \uu)^T \right) \\
%       \gamma(D\uu) &= \tfrac 1 2 D\uu \tcolon D\uu \\
%       \eta(\gamma) &= B(\theta,\dotsc)\big(\gamma_0 + \gamma \big)^{\frac{\mathfrak{p}-2}{2}} \\
%       \mathfrak{p} &= 1 + \tfrac{1}{\mathfrak{n}} \approx \tfrac 4 3 \\
%       T &= \bm 1 - \bm n \otimes \bm n \\
%     \end{align*}
% \end{column}
% \end{columns}
% \vspace{-1.5em}
%     with boundary conditions
%     \begin{align*}
%       (\eta D\bm u - p\bm 1)\cdot\bm n =
%       \begin{cases}\bm 0 & \text{free surface} \\
%         -\rho_w z \bm n & \text{ice-ocean interface}\end{cases} \\
%       \bm u = \bm 0\qquad\qquad \text{frozen bed}, \theta < \theta_0 \\
%       \left. \begin{aligned}
%           \bm u \cdot \bm n &= \bm g_{\text{melt}}(T\uu,\dotsc) \\
%           T (\eta D\bm u - p\bm 1)\cdot\bm n &= \bm g_{\text{slip}}(T \bm u,\dotsc) \end{aligned}\right\}
%       \text{nonlinear slip}, \theta \ge \theta_0 \\
%     \end{align*}
%     \vspace{-3em}
%     \[ \bm g_{\text{slip}}(T\uu) = \beta_{\mathfrak{m}}(\dotsc) \lvert T\bm u \rvert^{\mathfrak{m}-1} T \bm u \]
%     Navier $\mathfrak{m}=1$, \quad Weertman $\mathfrak{m}\approx \frac 1 3$, \quad Coulomb $\mathfrak{m}=0$.
% \end{frame}

\input{slides/GroundingLine/ALEFormulation.tex}

\begin{frame}{Stokes challenges}
  \begin{block}{Mass conservation is critical}
    \begin{itemize}
    \item Staggered grid finite difference (hard to deal with geometry)
    \item Stabilized methods (conservation artifacts when non-smooth)
    \item Inf-sup stable mixed finite element method
      \begin{itemize}
      \item Use discontinuous pressure to enforce local mass conservation
      \item Inf-sup constant decays like $\sqrt{\epsilon}$ for $Q_k-P_{k-1}^{\text{disc}}$
      \item Sub-optimal order of accuracy for $Q_k-Q_{k-2}^{\text{disc}}$
      \end{itemize}
    \end{itemize}      
  \end{block}
  \begin{block}{Solving saddle-point problems}
    \begin{itemize}
    \item Not uniformly elliptic: solvers are much less robust
    \item Standard preconditioners do not work
    \item Coupled multigrid with Vanka smoothers offer best performance,\\
      \quad not robust for stretched grids or anisotropic viscosity
    \item Block preconditioners require approximate commutators, \\
      \quad fragile for strong anisotropy and non-smooth viscosity
    \end{itemize}
  \end{block}
\end{frame}

\section[Slip]{Slip boundary conditions on bumpy surfaces}
\input{slides/GroundingLine/Normals.tex}

\input{slides/GroundingLine/ALEBlockForm.tex}

\section{Coupling}
\input{slides/PETSc/Coupling.tex}

\begin{frame}{Outlook}
  \begin{block}{}
    \begin{itemize}
    \item We have textbook multigrid efficiency for hydrostatic equations
    \item Finally a good algebraic interface for tightly-coupled multiphysics
    \item Technical challenges for Stokes
    \item Local conservation is critical, well-balanced slip
    \item Singularities: reentrant corners, transition from frozen \\ \quad 
      to slip bounadry conditions, grounded margins, grounding lines
    \item Stiff geometric coupling terms
    \item IMEX time integration: additive Runge-Kutta
    \end{itemize}    
  \end{block}
  \begin{block}{Tools}
    \begin{itemize}
    \item PETSc\ \url{http://mcs.anl.gov/petsc}
      \begin{itemize}\item ML, Hypre, MUMPS
      \end{itemize}
    \item ITAPS \url{http://itaps.org}
      \begin{itemize}\item MOAB, CGM, Lasso
      \end{itemize}
    \end{itemize}
  \end{block}
\end{frame}

\end{document}
